% !TEX root = ../main.tex
\documentclass[../main.tex]{subfiles}
\begin{document}

\section{Описание программной части разработанной системы}

В рамках данной работы было написано програмное обеспечение для работы с аппаратным обеспечением описаным выше. Программа обеспечение представляет собой кросплатформенное Desktop приложение, написанное на языке python с использованием графического фреймворка PyQt.

Чтобы запустить программу, предварительно нужно скачать и установить python для выбранной операционной системы. Установочные файлы и инструкции по установке есть на официальном сайте python.org. Также необходимо установить следующих набор библиотек c помощью комманд:

\begin{lstlisting}
pip install PyQt5
pip install numpy
pip install pyqtgraph
pip install pyserial
\end{lstlisting}

Даллее в коммандной строке необходимо перейти в директорию, где находится файл main.py. Теперь нужно подключить устройство к компьютеру через USB порт и запустить программу с помощью команды:

\begin{lstlisting}
python main.py
\end{lstlisting}


Программное обеспечение состоит из 2х основных компонент:
\begin{itemize}
    \item программа для микроконтроллера Arduino Due
    \item программа для компьютера
\end{itemize}

Перед тем, как рассматривать эти модули отдельно, рассмотрим файл \verb|main.py|. С начала в нем производится импорт следующих модулей:

создание тредов (модуль стандартной библиотеки)
\begin{lstlisting}
import threading       
\end{lstlisting}

общение между тредами с использованием сигналов (модуль стандартной библиотеки)
\begin{lstlisting}
import signal
\end{lstlisting}

используется для парсинга аргументов коммандной строки
\begin{lstlisting}
import sys
\end{lstlisting}

фреймворк для графического интерфейса
\begin{lstlisting}
import PyQt5.QtWidgets
\end{lstlisting}

модуль отвечающий за графический интерфейс программы
\begin{lstlisting}
import gui
\end{lstlisting}

сбор сигнала от устройства через usb(серийный порт)
\begin{lstlisting}
import serial_port
\end{lstlisting}

чтобы запустить приложение с испольованием \verb|PyQt5|, нужно создать объект \verb|QApplication| а также объект \verb|gui.GUI| - основной класс в котором описывается графический интерфейс:

\begin{lstlisting}
app = PyQt5.QtWidgets.QApplication(sys.argv)
gui = gui.GUI()
\end{lstlisting}

Чтобы приложение можно было правильно закрыть по нажатию Ctrl-C, регистрируется функцию которая будет обрабатываеть сигнал \verb|SIGINT| (Keyboard Interrupt, прерывание с клавиатуры):

\begin{lstlisting}
def ctrl_c_handler(sig, frame):
    serial_port.stop_flag = True
    app.quit()

signal.signal(signal.SIGINT, ctrl_c_handler)
\end{lstlisting}

при прерывании с клавиатуры переменная \verb|stop_flag| принимает значение \verb|True| что говорит модулю \verb|serial_port| что следует прекратить сбор данных с usb и закрыть соединение. Также производится выход из графического приложения.

модули \verb|serial_port| и \verb|gui| работают в разных потоках. Это нужно для того, чтобы сбор сигнала производился без пауз, а также чтобы графический интерфейс программы не зависал и быстро реагировал на действия пользователя.

Графический интерфейс будет испольняться в основном треде программы. Для моудуля \verb|serial_port| создается дополнительные тред:


\begin{lstlisting}
serial_reader = threading.Thread(
    target=serial_port.run,
    args=(gui.bmp_signal, gui.mic_signal)
)
\end{lstlisting}

\verb|target| - функция, которую будет исполнять тред, \verb|args| - аргументы функции - 2 Qt-сигнала, о которых будет сказано далее.

Рассмотрим программу для микроконтроллера Arduino а затем рассмотрим более детально, что происходит в модулях \verb|serial_port| и \verb|gui|

\newpage
\end{document}
