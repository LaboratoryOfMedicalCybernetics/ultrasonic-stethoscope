% !TEX root = ../main.tex
\documentclass[../main.tex]{subfiles}
\begin{document}
\thispagestyle{empty}
\begin{small}
\begin{center}
ОТЗЫВ
\end{center}

\noindent
на ВКР магистерскую диссертацию Родионова Александра Алексанровиче, обучающегося по направлению подготовки 02.04.01 Математика и компьютерные науки, компьютерное моделирование и искусственный интеллект на факультете компьютерных наук Воронежского государственного университета на тему

\begin{center}
\underline{<<Цифровая акустическая система оценки состояния лёгких>>}
\end{center}

Магистерская диссертация посвящена решению важной проблемы: разработке цифровой акустической системы оценки состояния лёгких.

В работе проведен сравнительный анализ существующих на рынке устройств. Были разобраны основные технические характеристики аналогов, были выявлены их достоинства и недостатки.

В теоретической части исследования была реализована компьютерная модель распространения звука в легких. Была выбрана подходящая математическая модель - волновое уравнение для неоднородной среды. Модель основывается на реальных данных - на компьютерных снимках легких. Модель реализована с помощью совремменных фреймворков в виде кроссплатформенного приложения.

В практической части было собрано устройство на базе микроконтроллера Arduino Due с использованием микрофона, динамиков и датчиков давления. В изготовлении устройсва использовалась 3D-печать.

При написании программного обеспечения для устройства были выбраны оптимальные для данной задачи технолонии: язык программирования python, обладающий большим количеством библиотек для анализа сигналов, визуализации данных. Графический фреймворк Qt позволяет легко создавать кроссплатформенные приложения.

В магистерской диссертации продемонстрированы хорошие аналитические способности, умение систематизировать и анализировать собранную информацию, делать предложения, обобщения и выводы.

Выбранная проблематика раскрыта полно и всесторонне. Магистерская диссертация заслуживает оценку "отлично".

\vspace{1cm}

\begin{flushleft}
{
Руководитель $~~~~~~~~~\underset{\text{\emph{подпись}}}{\rule[-1.6mm]{3cm}{0,25mm}}$ $\underset{\text{\emph{расшифровка подписи}}}{\underline{\phantom{aaaaaaaa}\text{Туровский Я.А. \phantom{aaaa.} к. мед. н., доцент}}}$
\\\vspace{1cm}
\underline{\phantom{aaa}}.\underline{\phantom{aaa}}.20\underline{\phantom{aaa}}

}
\end{flushleft}\! \! \! \! \! \! \! \!

\end{small}
\newpage
\end{document}
