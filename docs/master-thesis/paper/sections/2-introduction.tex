% !TEX root = ../main.tex
\documentclass[../main.tex]{subfiles}
\begin{document}
\section*{Введение}
\addcontentsline{toc}{section}{\protect\numberline{}Введение}%

Аускультация (выслушивание) звуков, исходящих от различных органов - одна из областей медицинской диагностики. Прибор для выслушивания звуков называется стетоскоп. Стетоскопы бывают акустические и электронные. Акустический стетоскоп передает звук от пациента непосредственно в ухо врачу. У электронных есть микрофон, который передает звук либо через динамики врачу, либо записывает для дальнейшего анализа.

Медицинская диагностика сердечно-сосудистых заболеваний является одной из важнейших отраслей современной медицины. По данным Всемирной Организации Здравоохранения (ВОЗ) сердечно-сосудистые заболевания - это основная причина смертей в мире. Максимальное количество смертей - это сердечно-сосудистые заболевания. Процентное сообношение этих смертей составляет примерно 31\% или в абсолютных значениях  17.5 млн. человек. Сердечно-сосудистые заболевания нуждаются в раннем обнаружении, диагностики, консультировании и лечении.

Медицинская диагностика заболеваний легочной системы является не менее важной. Заболевания легочной системы, такие как респираторные инфекции дыхательных путей, хроническая обструктивная болезнь лёгких или ХОБЛ, рак легких и туберкулез входят в 10 болезней, от которых чаще всего умирают люди. (статистика ВОЗ на 2015 год).

Обычно люди, страдающие заболеваниями сердечно-сосудистой системы или легочной системы обращаются к участковому врачу в поликлинику (первичное звено). В первичном звене врач принимает решение о дальнейших действиях по лечению больного. Проблема заключается в том что в первичном звене РФ почти нет автоматизации. Из за этого врач первичного звена может принять неправильные решение из за несовершенства оборудования. Эту проблему можно решить, обеспечив врачей первичного звена оборудованием, позволяющем очень точно и автоматически определять состояние пациента и помогать принять однозначное решение о дальнейшем лечении пациента. Основным инструментом диагностики у врачей первичного звена РФ является стетоскоп, поэтому совершенствование стетоскопов должно быть основным трендом в развитии диагностики первичного звена. Совершенствование должно вестись не только в сторону улучшения обработки слышимого звука, но также и в сторону новых частотных диапазонов (ультразвука).

Проблемой существующих на рынке электронных стетоскопов является невысокое качество звука. Под невысоким качеством звука подразумевается низкая частота дискретизации получаемого на выходе сигнала и как следствие неспособность выдавать данные о высокочастотном диапазоне звука (в частности ультразвука). Эти стетоскопы имеют частоту, не охватывающюю весь звуковой диапазон, слышимый человеком, следовательно они теряют массу информации, которая может быть полезна врачам для осуществления медицинской диагностики пациентов.
\newpage
\end{document}
