Введение

Аускультация (выслушивание) звуков, исходящих от различных органов - одна из областей медицинской диагностики. Существуют приборы пассивной аускультации - стетоскопы. Стетоскопы бывают акустические и электронные. Акустический стетоскоп передает звук от пациента непосредственно в ухо врачу. У электронных есть микрофон, который передает звук либо через динамики врачу, либо оцифровывает сигнал и передает на компьютер для дальнейшего анализа.

Кроме пассивной, существует также активная аускультаця. При активной аускультации через легкие пациента проходит звук заданной частоты. Звук проходя через легкие подвергается искажениям и регистрируется. В данной работе представлена информационная система позволяющая производить как пассивную так и активную аускультацию легких.

% Таким образом может строиться карта легких. Но только у нас пока карта легких не строится.

Активная аускультация позволяет получать новую информацию, которую нельзя получить при помощи пассивной аускультации. С помощью активной аускультации можно построить карту легких пациента и сравнивать ее с картой легких здорового человека.


Заключение

В результате данной работы была создана система активной и пассивной аускультации легких. Устройство позволяет анализировать в реальном времени звуковой сигнал высокого качества Также было написано програмное обеспечение к этому прибору. Есть возможность рассмотреть различные характеристики сигнала, такие как спектр Фурье. Также есть возможность записывать аудиосигнал на жесткий диск для его последующей обработки. Сигнал записывается с частотой дискретизации 660кГц. Данная частота позволяет анализировать ультразвуковой сигнал до 330кГц.

Данное устройство предназначено для получения дополнительной информации из звука поступающего от сердца, легких и других внутренних органов, которую нельзя услышать на обычном стетоскопе. (методоми пассивной аускультации) Данная дополнительная информация может быть использована врачом для осуществления более качественной медицинской диагностики.

- Создана аппаратная часть устройства состоящая из микроконтроллера, микрофона, динамиков, датчиков давления и компьютера.
- Обеспечен диапазон принятия звуковых сигналов в диапазоне до 40кГц
- Написано програмное обеспечение для микроконтроллера Arduino
- Обеспечена обработка полученного сигнала с использованием спектральных методов оценивания
- Написано програмное обеспечения для компьютера
- Обеспечена возможность обработки сигнала при помощи технологии Nvidia CUDA
- Разработана компьютерная моделб распространения звука в легких (двухмерный и трехмерный варианты)
