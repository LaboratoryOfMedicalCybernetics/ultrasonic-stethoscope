% !TEX root = ../paper.tex
\documentclass[../paper.tex]{subfiles}

\begin{document}

\newpage
\section*{\centering Реферат}

\par\noindent Бакалаврская работа 50с., 32 рисунка, 25 источников
% \vspace{0.5cm}

МЕДИЦИНСКАЯ ДИАГНОСТИКА, СТЕТОСКОП, АНАЛОГО ЦИФРОВЫЕ ПРЕОБРАЗОВАТЕЛИ, ЦИФРОВАЯ ОБРАБОТКА СИГНАЛОВ, ВИЗУАЛИЗАЦИЯ ЗВУКОВОГО СИГНАЛА, БЫСТРОЕ ПРЕОБРАЗОВАНИЕ ФУРЬЕ

Целью данной работы является создание доступного и простого в производстве широкополосного цифрового стетоскопа. Стетоскоп должен быть способен регистрировать высокое качество звука. Он должен иметь высокую частоту дискретизации (больше 100кГц) и способен регистрировать сигнал в ультразвуковом диапозоне.(больше 20кГц)

В процессе выполнения работы тестировались 3 различных аналого цифровых преобразователя. Были выявлены плюсы и минусы каждого из АЦП применительно к поставленной задаче. Также был создан усилитель сигнала для микрофона.

В результате работы был создан рабочий прототип цифрового широкополосного стетоскопа. Стетоскоп способен регистрировать сигнал с частотой дискретизации 660кГц и способен анализировать ультразвук до 100кГц. Также было написано програмное обеспечения для этого стетоскопа, позволяющего визуализировать сигнал, обрабатывать сигнал с помощью быстрого преобразования фурье (БПФ), а также визуализировать сигнал после обработки БПФ. В возможности програмного обеспечения входит обработка сигнала на удаленном высокопроизводительном сервере с помощью технологии Nvidia CUDA.

\clearpage
\normalsize

\end{document}
