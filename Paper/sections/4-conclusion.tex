% !TEX root = ../paper.tex
\documentclass[../paper.tex]{subfiles}
\begin{document}

В результате данной работы был создан рабочий прототип устройства, позволяющего анализировать в реальном времени звуковой сигнал высокого качества. Сигнал записывается с частотой дискретизации 660кГц. Данная частота позволяет анализировать ультразвуковой сигнал до 330кГц.

Данное устройство предназначено для получения дополнительной информации из звука поступающего от сердца, легких и других внутренних органов, которую нельзя услышать на обычном стетоскопе. Данная дополнительная информация может быть использована врачом для осуществления более качественной медицинской диагностики.

Сфера применения не ограничена медициной, устройство позволяет анализировать любые типы ультразвуковых сигналов.

Развитие проекта можно продолжить в направлении улучшения качества звука. Для этого нужно использовать более дорогие микрофон и усилитель. Также нужно произвести опрос врачей о том, какие именно характеристики звука со стетоскопа важны.

Улучшений в програмном обеспечении можно достигнуть путем оптимизации алгоритмов распаралеливания на нескольких ядрах процессора или на видеокарте. Также, на основе информации от докторов, можно сделать систему распознавания различных забовалеваний легких и сердца.

Разработка данного проекта велась с помощью системы контроля версий git. Исходный код програмного обеспечения, этапы создания и документация доступны по адресу:

\url{https://github.com/tandav/ultrasonic-stethoscope}

\end{document}
