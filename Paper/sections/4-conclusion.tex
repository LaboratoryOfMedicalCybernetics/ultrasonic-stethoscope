% !TEX root = ../paper.tex
\documentclass[../paper.tex]{subfiles}
\begin{document}

В процессе данной работы был создан рабочий прототип устройства, позволяющего анализировать в реальном времени звуковой сигнал, поступающий от сердца или легких. Данное устройство может применяться для получения дополнительной информации, которую человек не может услышать на обычном стетоскопе.

Сфера применения не ограничена медициной, устройство позволяет анализировать любые типы ультразвуковых сигналов.

Развитие проекта можно продолжить в направлении улучшения качества звука. Для этого нужно использовать более дорогие микрофон и усилитель. Также нужно произвести опрос врачей о том, какие именно характеристики звука со стетоскопа важны.

Улучшений в програмном обеспечении можно достигнуть путем оптимизации алгоритмов распаралеливания на нескольких ядрах процессора или на видеокарте. На основе информации от докторов, можно сделать систему распознавания различных забовалеваний легких и сердца.

Разработка данного проекта велась с помощью системы контроля версий git. Исходный код програмного обеспечения, этапы создания и документация доступны по адресу:

\url{https://github.com/tandav/ultrasonic-stethoscope}

\end{document}
