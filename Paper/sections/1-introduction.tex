% !TEX root = ../paper.tex
\documentclass[../paper.tex]{subfiles}
\begin{document}

\subsection{Постановка проблемы}
Аускультация (выслушивание) звуков, исходящих от различных органов - одна из областей медицинской диагностики. Прибор для выслушивания звуков называется стетоскоп. Стетоскопы бывают акустические и электронные. Акустический стетоскоп передает звук от пациента непосредственно в ухо врачу. У электронных есть микрофон, который передает звук либо через динамики врачу, либо записывает для дальнейшего анализа.

Проблемой существующих на рынке электронных стетоскопов является невысокое качество звука. Под невысоким качеством звука подразумевается низкая частота дискретизации получаемого на выходе сигнала и как следствие неспособность выдавать данные о высокочастотном диапазоне звука (в частности ультразвука).

Эти стетоскопы теряют массу информации, которая может быть полезна врачам для осуществления медицинской диагностики пациентов.

\subsection{Цель}
Целью данной курсовой работы является создание доступного и простого в производстве цифрового стетоскопа. Стетоскоп должен быть способен регистрировать высокое качество звука. Он должен иметь высокую частоту дискретизации (600 кГц) и способен регистрировать сигнал в ультразвуковом диапозоне (до 100-300кГц)

В данной работе описывается создание ультразвукогого стетоскопа.

В ходе работы был создан рабочий прототип прибора. Прибор позволяет получать сигнал от ультразвукового микрофона. Также было написано програмное обеспечение к этому прибору. Програмное обеспечение позволяет позволяет записывать и анализировать звуковые сигналы в реальном времени. Есть возможность рассмотреть различные характеристики сигнала, такие как спектр Фурье, скользящее среднее. Также есть возможность записывать аудиосигнал на жесткий диск для его последующей обработки.

\subsection{Применение прибора на практике}
Устройство планируется использоваться в медицине: анализ ультразвуковой составляющей звука от сердца и легких.

Данный прибор может использоваться в медицине для получения и анализа сигнала высокого качества. Например звук сердца и лёгких. 

Это устройство поможет лучше анализировать звук сердца. С помошью визуализации сигнала можно получить больше информации о звуке внутренних органов, чем простое прослушивание.

Также прибор можно использовать в других областях, где необходим анализ ультразвука. 

Например изучение дельфинов или летучих мышей. 

Сердечно-сосудистые заболевания - самая распространенная причина смерти в мире по данным Всемирной Организации Здравоохранения (ВОЗ). Также распространенными являются заболевания легких. Своевременное наблюдение за состоянием сердца и легких, и обнаружение заболеваний - важная задача здравоохранения.
\end{document}
