\documentclass[14pt]{extarticle}
\usepackage[russian]{babel}
\usepackage[utf8x]{inputenc}
\usepackage{float}
\usepackage{amsmath}
\usepackage{graphicx}
\usepackage{subcaption}
\usepackage[colorinlistoftodos]{todonotes}
\renewcommand{\baselinestretch}{1.5}
\usepackage[left=3cm,right=1cm,top=1.5cm,bottom=2cm]{geometry}
\setlength{\parindent}{0mm}
\setlength{\parskip}{1em}
\usepackage{hyperref}
\usepackage{subfiles}
\hypersetup{
    colorlinks,
    citecolor=black,
    filecolor=black,
    linkcolor=black,
    urlcolor=black
}

\begin{document}

\begin{titlepage}
\subfile{sections/0-titlepage}
\end{titlepage}

\tableofcontents
\newpage 

\section{Введение}
\subfile{sections/1-introduction}
\newpage  

\section{Описание аппаратной части разработанной системы}
\subfile{sections/2-hardware}
\newpage

\section{Описание программного обеспечения}
\subfile{sections/3-software}
\newpage

\section{Заключение}
В процессе данной работы был создан рабочий прототип устройства, позволяющего анализировать в реальном времени звуковой сигнал, поступающий от сердца или легких. Данное устройство может применяться для получения дополнительной информации, которую человек не может услышать на обычном стетоскопе.

Сфера применения не ограничена медициной, устройство позволяет анализировать любые типы ультразвуковых сигналов.

Развитие проекта можно продолжить в направлении улучшения качества звука. Для этого нужно использовать более дорогие микрофон и усилитель. Также нужно произвести опрос врачей о том, какие именно характеристики звука со стетоскопа важны.

Улучшений в програмном обеспечении можно достигнуть путем оптимизации алгоритмов распаралеливания на нескольких ядрах процессора или на видеокарте. На основе информации от докторов, можно сделать систему распознавания различных забовалеваний легких и сердца.

Разработка данного проекта велась с помощью системы контроля версий git. Исходный код програмного обеспечения, этапы создания и документация доступны по адресу:

\url{https://github.com/tandav/ultrasonic-stethoscope}

\newpage
\renewcommand\refname{Ссылки на источники}
\begin{thebibliography}{}
\bibitem{latexcompanion} 
Аналого цифровой преобразователь ЛА-н10-12USB\\
\url{http://www.rudshel.ru/show.php?dev=14}

\bibitem{latexcompanion} 
Драйверы и програмное обеспечение для устройств ЗАО "Руднев-Шиляев"\\
\url{http://rudshel.ru/software.html}

\bibitem{latexcompanion} 
Документация по программированию устройств ЗАО "Руднев-Шиляев"\\
\url{http://www.rudshel.ru/soft/SDK2/Doc/CPP_USER_RU/html/index.html}

\bibitem{latexcompanion} 
Руководство пользователя ЛА-н10-12USB\\
\url{http://www.rudshel.ru/pdf/LA-n10-12USB(y).rar}

\bibitem{latexcompanion} 
Внешний USB АЦП/ЦАП E14-140-M\\
\url{http://www.lcard.ru/products/external/e-140m)}

\bibitem{latexcompanion} 
Схема усилителя для микрофона\\
\url{http://full-chip.net/analogovaya-elektronika/70-usilitel-dlya-elektretnogo-mikrofona-s-nizkim-urovnem-shuma-shema.html}

\bibitem{latexcompanion} 
Руководство пользователя и технические характеристики операционного усилителя MCP6022\\
\url{https://lib.chipdip.ru/291/DOC000291231.pdf}

\bibitem{latexcompanion} 
Исходный код, документация и этапы создания проекта ультразвукового стетоскопа\\
\url{https://github.com/tandav/ultrasonic-stethoscope}

\end{thebibliography}


\end{document}
