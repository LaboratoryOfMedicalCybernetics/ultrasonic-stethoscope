%%%%%%%%%%%%%%%%%%%%%%%%%%%%%%%%%%%%%%%%%
% University Assignment Title Page 
% LaTeX Template
% Version 1.0 (27/12/12)
%
% This template has been downloaded from:
% http://www.LaTeXTemplates.com
%
% Original author:
% WikiBooks (http://en.wikibooks.org/wiki/LaTeX/Title_Creation)
%
% License:
% CC BY-NC-SA 3.0 (http://creativecommons.org/licenses/by-nc-sa/3.0/)
% 
% Instructions for using this template:
% This title page is capable of being compiled as is. This is not useful for 
% including it in another document. To do this, you have two options: 
%
% 1) Copy/paste everything between \begin{document} and \end{document} 
% starting at \begin{titlepage} and paste this into another LaTeX file where you 
% want your title page.
% OR
% 2) Remove everything outside the \begin{titlepage} and \end{titlepage} and 
% move this file to the same directory as the LaTeX file you wish to add it to. 
% Then add \input{./title_page_1.tex} to your LaTeX file where you want your
% title page.
%
%%%%%%%%%%%%%%%%%%%%%%%%%%%%%%%%%%%%%%%%%
%\title{Title page with logo}
%----------------------------------------------------------------------------------------
%	PACKAGES AND OTHER DOCUMENT CONFIGURATIONS
%----------------------------------------------------------------------------------------

% \documentclass[14pt]{extarticle}
\documentclass[bibliography=totocnumbered]{scrartcl}
% \usepackage[english]{babel}
\usepackage[russian]{babel}
\usepackage[utf8x]{inputenc}
\usepackage{amsmath}
\usepackage{graphicx}
\usepackage[colorinlistoftodos]{todonotes}
\usepackage{geometry}
\begin{document}


\begin{titlepage}
\newgeometry{margin=2cm}
\newcommand{\HRule}{\rule{\linewidth}{0.5mm}} % Defines a new command for the horizontal lines, change thickness here

\center % Center everything on the page
 
%----------------------------------------------------------------------------------------
%	HEADING SECTIONS
%----------------------------------------------------------------------------------------
\textsc {
\footnotesize{
минобрнауки россии\\
федеральное государственное бюджетное образовательное учреждение\\
высшего профессионального образования}\\
\large{Воронежский государственный университет}
}\\[1.0cm] % Name of your university/college


\textsc{\largeФакультет компьютерных наук}\\ % Major heading such as course name
\textsc{\footnotesize010200 Математика и компьютерные науки}\\[1.0cm] 
\textsc{\Large дипломная работа}\\[0.5cm] % Minor heading such as course title


%----------------------------------------------------------------------------------------
%	TITLE SECTION
%----------------------------------------------------------------------------------------

\HRule \\[0.4cm]
{ \huge \bfseries Ультразвуковой стетоскоп}\\[0.4cm] % Title of your document
\HRule \\[1.5cm]
 
%----------------------------------------------------------------------------------------
%	AUTHOR SECTION
%----------------------------------------------------------------------------------------


\begin{flushleft} \large
\emph{Зав. кафедрой:} С.Д. \textsc{Кургалин}, д. ф-м н., проф.\\
\emph{Студент:} А.А. \textsc{Родионов}, 3 курс, гр 6.1 \\ % Your name
\emph{Руководитель:} Я.А. \textsc{Туровский}, к. мед. н , доцент % Supervisor's Name
\end{flushleft}


% If you don't want a supervisor, uncomment the two lines below and remove the section above
% \Large \emph{Author:}\\
% John \textsc{Smith}\\[3cm] % Your name

%----------------------------------------------------------------------------------------
%	DATE SECTION
%----------------------------------------------------------------------------------------
\vfill % Fill the rest of the page with whitespace
\begin{center}
Воронеж 2016
\end{center}
\end{titlepage}

\tableofcontents
\newpage 
\section{Введение}
Целью данной курсовой работы является создание доступного и простого в производстве цифрового стетоскопа, способного организовать прослушивание легких и сердца человека.

Особенностью данного стетоскопа является то что он может регистрировать сигнал в ультразвуковом диапазоне: до 100 кГц.

В данной работе описывается создание ультразвукогого стетоскопа.

В ходе дипломной работы был создан рабочий прототип прибора. Прибор позволяет получать сигнал от ультразвукового микрофона. Также было написано програмное обеспечение к этому прибору. Програмное обеспечение позволяет позволяет записывать и анализировать звуковые сигналы в реальном времени. Есть возможность рассмотреть различные характеристики сигнала, такие как спектр Фурье, скользящее среднее. Также есть возможность записывать аудиосигнал на жесткий диск для его последующей обработки.

\subsection{Применение прибора на практике}
Может юзаться в медицине: легкие, 

Данный прибор может использоваться в медицине для получения и анализа сигнала высокого качества. Например звук сердца и лёгких. 

Это устройство поможет лучше анализировать звук сердца. С помошью визуализации сигнала можно получить больше информации о звуке внутренних органов, чем простое прослушивание.

Также прибор можно использовать в других областях, где необходим анализ ультразвука. 

Например изучение дельфинов или летучих мышей. 

Сердечно-сосудистые заболевания - самая распространенная причина смерти в мире по данным Всемирной Организации Здравоохранения (ВОЗ). Также распространенными являются заболевания легких. Своевременное наблюдение за состоянием сердца и легких, и обнаружение заболеваний - важная задача здравоохранения.

\newpage 
\section{Цель работы}

Целью данной курсовой работы является создание доступного и простого в производстве цифрового стетоскопа способного организовать прослушивание легких и сердца человека. Данный цифровой стетоскоп работает в паре с компьютерной программой. Программа позволяет просматривать аудиосигнал в визуальном виде. Также можно увидеть спектр данного сигнала, полученный с помощью преобразования Фурье аудиосигнала со стетоскопа.

Это устройство поможет лучше анализировать звук сердца. С помошью визуализации сигнала можно получить больше информации о звуке внутренних органов, чем простое прослушивание.
\newpage 
\section{Описание устройства}
Устройство состоит из нескольких частей соединенных между собой. От аналогового стетоскопа берется мембрана и соединительная трубка. С одной стороны к соединительной трубке подсоединяется мембрана, с другой - микрофон. Сигнал с микрофона подается на усилитель. С усилителя сигнал подается на Аналогово-Цифровой-Преобразователь (АЦП). Аналогово-Цифровой-Преобразователь подключается к компьютеру через USB-порт.

\begin{center}
Краткая схема прибора: \\[0.4cm]

\noindent\scriptsize{{Мембрана → Соединительная Трубка → Микрофон → Усилитель → АЦП → Компьютер}}
\end{center}
\label{sec:examples}

\subsection{Выбранный микрофон}

В качестве микрофона был выбран SWEN MK-200. \\

\begin{table}[h]
\centering
\caption{Технические характеристики SWEN MK-200}
\label{my-label}
\begin{tabular}{|l|l|}
\hline
Чувствительность, дБ           & -60 ± 3                    \\ \hline
Диапазон частот, Гц            & 50 – 16 000                \\ \hline
Размер микрофонного модуля, мм & 9×7                        \\ \hline
Тип разъема                    & мини-джек Ø 3,5 мм (3 pin) \\ \hline
Длина кабеля, м                & 1,8                        \\ \hline
Вес, г                         & 63                         \\ \hline
\end{tabular}
\end{table}

\subsection{Выбранный Усилитель}

Lorem Ipsum is simply dummy text of the printing and typesetting industry. Lorem Ipsum has been the industry's standard dummy text ever since the 1500s, when an unknown printer took a galley of type and scrambled it to make a type specimen book. It has survived not only five centuries, but also the leap into electronic typesetting, remaining essentially unchanged. It was popularised in the 1960s with the release of Letraset sheets containing Lorem Ipsum passages, and more recently with desktop publishing software like Aldus PageMaker including versions of Lorem Ipsum.

\subsection{Выбранный АЦП}

Lorem Ipsum is simply dummy text of the printing and typesetting industry. Lorem Ipsum has been the industry's standard dummy text ever since the 1500s, when an unknown printer took a galley of type and scrambled it to make a type specimen book. It has survived not only five centuries, but also the leap into electronic typesetting, remaining essentially unchanged. It was popularised in the 1960s with the release of Letraset sheets containing Lorem Ipsum passages, and more recently with desktop publishing software like Aldus PageMaker including versions of Lorem Ipsum.


% % Commands to include a figure:
% \begin{figure}
% \centering
% \includegraphics[width=0.5\textwidth]{frog.jpg}
% \caption{\label{fig:frog}This is a figure caption.}
% \end{figure}


\newpage
\section{Описание компьютерной программы}
\subsection{О программе}
Компьютерная программа для обработки сигнала со стетоскопа написана на языке python c использованием библиотеки pyaudio (для работы со звуком) и библиотеки matplotlib (для визуализации сигналов с помощью графиков)
\subsection{GUI Parameters}
В данном участке кода можно выбрать параметры отображения Графического Интерфейса Пользователя (Graphical User Interface / GUI)
Представлено три параметра:
\begin{verbatim}
timeDomain
freqDomain
lpcOverlay
\end{verbatim}

Каждый из параметров может принимать значение \verb|True| или \verb|False|. В соответсвии с этими параметрами на экране будут(\verb|True|) или не будут(\verb|False|) отображаться соответсвующие элементы.

Параметр \verb|timeDomain| отвечает за отображение звуковой волны. (Зависимость амплитуды сигнала от времени). 

Параметр \verb|freqDomain| отвечает за отображение спектра Фурье звукового сигнала. (Зависимость амплитуды сигнала от частоты). 

Параметр \verb|lpcOverlay| отвечает за отображение усредненного спектра Фурье звукового сигнала. (Апроксимация спектра Фурье многочленами)

\subsection{Stream Parameters}
В данной секции можно настроить параметры захвата аудио с микрофона. Представлено шесть параметров настройки:
\begin{verbatim}
DEVICE
CHUNK
WINDOW
FORMAT
CHANNELS
RATE
\end{verbatim}

Параметр \verb|DEVICE| позволяет выбрать входной порт аудиокарты. Значение этого параметра по умолчанию 0. Если у аудиокарты портов много, то необходимо указать соответсвуующее выбранному  порту значение этого парметра. 

Параметр \verb|CHUNK| - это размер блока сигнала, захватываемого программой за одну итерацию программы. 

Параметр \verb|WINDOW| отвечает за ширину окна программы и количество значений, отображаемых в окне.

Параметр \verb|FORMAT| позволяет выбрать тип данных аудиосигнала. Поддерживаются значения  \verb|paFloat32, paInt32, paInt24, paInt16, paInt8, paUInt8|.

Параметр \verb|CHANNELS| отвечает количество записываемых каналов. Может принимать значения 1 или 2. 

Параметр \verb|RATE| позволяет выбрать частоту дискретизации входного сигнала в Герцах.

\subsection{Spectral parameters}
В данном участке кода представлено два параметра:
\begin{verbatim}
ORDER
NFFT
\end{verbatim}
Параметр \verb|ORDER| отвечает за порядок многочлена, который апроксимирует Фурье-спектр сигнала.

Параметр \verb|NFFT| - размер Фурье спектра по горизонтальной оси.
\subsection{Порядок выполнения программы}
Сначала создается аудиопоток на основе значений Stream Parameters. Затем на основе параметров GUI создаютсе те элементы UI, которые были выбраны пользователем. В этом участке кода можно также уточнить некоторые особенности отображения формы, такие, как наличие сетки (\verb| plt.grid()|) на графике, настройки цвета, количество отображаемых чисел на осях графика. Также можно выбрать границы значений по осям OX и OY для графиков. Также, при необходимости перевести какую-то из осей в логарифмический формат, нужно добавить  соответсвенно строчки:
\begin{verbatim}
plt.xscale('log')
plt.yscale('log')
\end{verbatim}

После инициализации формы с задаными настройками запускается функция анимации: 
\begin{verbatim}
animation = FuncAnimation(fig, update, interval=10)
\end{verbatim}
В параметрах этой функции можно задать временной интервал между кадрами в милисекундах. Функция \verb|animation| запускает функцию \verb|update|. Эта функция обновляет график на каждом кадре анимации. Каждый кадр считывается очередной блок аудиосигнала (\verb|CHUNK|) и на его основе строятся графики, выбранные пользователем. 

Чтобы построить график спектра Фурье сигнала, в функции \verb|spectral_estimate| выполняется преобразование фурье. Для этой цели используется библиотека numpy.

Чтобы построить апроксимированный многочленом спектр Фурье, выполняется функция \verb|lpc_spectrum|.

\section{Заключение}
\newpage

\begin{thebibliography}{9}
\bibitem{latexcompanion} 
Легочный Фонд России
\\\texttt{http://legkie.org}
 
\bibitem{einstein} 
Albert Einstein. 
\textit{Zur Elektrodynamik bewegter K{\"o}rper}. (German) 
[\textit{On the electrodynamics of moving bodies}]. 
Annalen der Physik, 322(10):891–921, 1905.
 
\bibitem{knuthwebsite} 
Knuth: Computers and Typesetting,
\\\texttt{http://www-cs-faculty.stanford.edu/\~{}uno/abcde.html}
\end{thebibliography}


\end{document}
